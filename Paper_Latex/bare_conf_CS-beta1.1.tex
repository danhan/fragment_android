
%% This style is provided exclusively for the ICSE 2012 main conference,
%% ICSE 2012 co-located events, and ICSE 2012 workshops.

%% bare_conf_ICSE12.tex
%% V1.4
%% 2012-01-21
%%

%% This is a skeleton file demonstrating the use of IEEEtran.cls
%% (requires IEEEtran.cls version 1.7 or later) with an IEEE conference paper.
%%
%% Support sites:
%% http://www.michaelshell.org/tex/ieeetran/
%% http://www.ctan.org/tex-archive/macros/latex/contrib/IEEEtran/
%% and
%% http://www.ieee.org/

%%*************************************************************************
%% Legal Notice:
%% This code is offered as-is without any warranty either expressed or
%% implied; without even the implied warranty of MERCHANTABILITY or
%% FITNESS FOR A PARTICULAR PURPOSE! 
%% User assumes all risk.
%% In no event shall IEEE or any contributor to this code be liable for
%% any damages or losses, including, but not limited to, incidental,
%% consequential, or any other damages, resulting from the use or misuse
%% of any information contained here.
%%
%% All comments are the opinions of their respective authors and are not
%% necessarily endorsed by the IEEE.
%%
%% This work is distributed under the LaTeX Project Public License (LPPL)
%% ( http://www.latex-project.org/ ) version 1.3, and may be freely used,
%% distributed and modified. A copy of the LPPL, version 1.3, is included
%% in the base LaTeX documentation of all distributions of LaTeX released
%% 2003/12/01 or later.
%% Retain all contribution notices and credits.
%% ** Modified files should be clearly indicated as such, including  **
%% ** renaming them and changing author support contact information. **
%%
%% File list of work: IEEEtran.cls, IEEEtran_HOWTO.pdf, bare_adv.tex,
%%                    bare_conf.tex, bare_jrnl.tex, bare_jrnl_compsoc.tex
%%*************************************************************************

% *** Authors should verify (and, if needed, correct) their LaTeX system  ***
% *** with the testflow diagnostic prior to trusting their LaTeX platform ***
% *** with production work. IEEE's font choices can trigger bugs that do  ***
% *** not appear when using other class files.                            ***
% The testflow support page is at:
% http://www.michaelshell.org/tex/testflow/



% Note that the a4paper option is mainly intended so that authors in
% countries using A4 can easily print to A4 and see how their papers will
% look in print - the typesetting of the document will not typically be
% affected with changes in paper size (but the bottom and side margins will).
% Use the testflow package mentioned above to verify correct handling of
% both paper sizes by the user's LaTeX system.
%
% Also note that the "draftcls" or "draftclsnofoot", not "draft", option
% should be used if it is desired that the figures are to be displayed in
% draft mode.
%
\documentclass[10pt, conference, compsocconf]{IEEEtran}
% Add the compsocconf option for Computer Society conferences.
%
% If IEEEtran.cls has not been installed into the LaTeX system files,
% manually specify the path to it like:
% \documentclass[conference]{../sty/IEEEtran}

\usepackage{graphicx,amssymb,amstext,amsmath}
\usepackage{balance}
\usepackage{hyperref}
\usepackage{multirow}


% Some very useful LaTeX packages include:
% (uncomment the ones you want to load)


% *** MISC UTILITY PACKAGES ***
%
%\usepackage{ifpdf}
% Heiko Oberdiek's ifpdf.sty is very useful if you need conditional
% compilation based on whether the output is pdf or dvi.
% usage:
% \ifpdf
%   % pdf code
% \else
%   % dvi code
% \fi
% The latest version of ifpdf.sty can be obtained from:
% http://www.ctan.org/tex-archive/macros/latex/contrib/oberdiek/
% Also, note that IEEEtran.cls V1.7 and later provides a builtin
% \ifCLASSINFOpdf conditional that works the same way.
% When switching from latex to pdflatex and vice-versa, the compiler may
% have to be run twice to clear warning/error messages.






% *** CITATION PACKAGES ***
%
%\usepackage{cite}
% cite.sty was written by Donald Arseneau
% V1.6 and later of IEEEtran pre-defines the format of the cite.sty package
% \cite{} output to follow that of IEEE. Loading the cite package will
% result in citation numbers being automatically sorted and properly
% "compressed/ranged". e.g., [1], [9], [2], [7], [5], [6] without using
% cite.sty will become [1], [2], [5]--[7], [9] using cite.sty. cite.sty's
% \cite will automatically add leading space, if needed. Use cite.sty's
% noadjust option (cite.sty V3.8 and later) if you want to turn this off.
% cite.sty is already installed on most LaTeX systems. Be sure and use
% version 4.0 (2003-05-27) and later if using hyperref.sty. cite.sty does
% not currently provide for hyperlinked citations.
% The latest version can be obtained at:
% http://www.ctan.org/tex-archive/macros/latex/contrib/cite/
% The documentation is contained in the cite.sty file itself.






% *** GRAPHICS RELATED PACKAGES ***
%
\ifCLASSINFOpdf
  % \usepackage[pdftex]{graphicx}
  % declare the path(s) where your graphic files are
  % \graphicspath{{../pdf/}{../jpeg/}}
  % and their extensions so you won't have to specify these with
  % every instance of \includegraphics
  % \DeclareGraphicsExtensions{.pdf,.jpeg,.png}
\else
  % or other class option (dvipsone, dvipdf, if not using dvips). graphicx
  % will default to the driver specified in the system graphics.cfg if no
  % driver is specified.
  % \usepackage[dvips]{graphicx}
  % declare the path(s) where your graphic files are
  % \graphicspath{{../eps/}}
  % and their extensions so you won't have to specify these with
  % every instance of \includegraphics
  % \DeclareGraphicsExtensions{.eps}
\fi
% graphicx was written by David Carlisle and Sebastian Rahtz. It is
% required if you want graphics, photos, etc. graphicx.sty is already
% installed on most LaTeX systems. The latest version and documentation can
% be obtained at: 
% http://www.ctan.org/tex-archive/macros/latex/required/graphics/
% Another good source of documentation is "Using Imported Graphics in
% LaTeX2e" by Keith Reckdahl which can be found as epslatex.ps or
% epslatex.pdf at: http://www.ctan.org/tex-archive/info/
%
% latex, and pdflatex in dvi mode, support graphics in encapsulated
% postscript (.eps) format. pdflatex in pdf mode supports graphics
% in .pdf, .jpeg, .png and .mps (metapost) formats. Users should ensure
% that all non-photo figures use a vector format (.eps, .pdf, .mps) and
% not a bitmapped formats (.jpeg, .png). IEEE frowns on bitmapped formats
% which can result in "jaggedy"/blurry rendering of lines and letters as
% well as large increases in file sizes.
%
% You can find documentation about the pdfTeX application at:
% http://www.tug.org/applications/pdftex





% *** MATH PACKAGES ***
%
%\usepackage[cmex10]{amsmath}
% A popular package from the American Mathematical Society that provides
% many useful and powerful commands for dealing with mathematics. If using
% it, be sure to load this package with the cmex10 option to ensure that
% only type 1 fonts will utilized at all point sizes. Without this option,
% it is possible that some math symbols, particularly those within
% footnotes, will be rendered in bitmap form which will result in a
% document that can not be IEEE Xplore compliant!
%
% Also, note that the amsmath package sets \interdisplaylinepenalty to 10000
% thus preventing page breaks from occurring within multiline equations. Use:
%\interdisplaylinepenalty=2500
% after loading amsmath to restore such page breaks as IEEEtran.cls normally
% does. amsmath.sty is already installed on most LaTeX systems. The latest
% version and documentation can be obtained at:
% http://www.ctan.org/tex-archive/macros/latex/required/amslatex/math/





% *** SPECIALIZED LIST PACKAGES ***
%
%\usepackage{algorithmic}
% algorithmic.sty was written by Peter Williams and Rogerio Brito.
% This package provides an algorithmic environment fo describing algorithms.
% You can use the algorithmic environment in-text or within a figure
% environment to provide for a floating algorithm. Do NOT use the algorithm
% floating environment provided by algorithm.sty (by the same authors) or
% algorithm2e.sty (by Christophe Fiorio) as IEEE does not use dedicated
% algorithm float types and packages that provide these will not provide
% correct IEEE style captions. The latest version and documentation of
% algorithmic.sty can be obtained at:
% http://www.ctan.org/tex-archive/macros/latex/contrib/algorithms/
% There is also a support site at:
% http://algorithms.berlios.de/index.html
% Also of interest may be the (relatively newer and more customizable)
% algorithmicx.sty package by Szasz Janos:
% http://www.ctan.org/tex-archive/macros/latex/contrib/algorithmicx/




% *** ALIGNMENT PACKAGES ***
%
%\usepackage{array}
% Frank Mittelbach's and David Carlisle's array.sty patches and improves
% the standard LaTeX2e array and tabular environments to provide better
% appearance and additional user controls. As the default LaTeX2e table
% generation code is lacking to the point of almost being broken with
% respect to the quality of the end results, all users are strongly
% advised to use an enhanced (at the very least that provided by array.sty)
% set of table tools. array.sty is already installed on most systems. The
% latest version and documentation can be obtained at:
% http://www.ctan.org/tex-archive/macros/latex/required/tools/


%\usepackage{mdwmath}
%\usepackage{mdwtab}
% Also highly recommended is Mark Wooding's extremely powerful MDW tools,
% especially mdwmath.sty and mdwtab.sty which are used to format equations
% and tables, respectively. The MDWtools set is already installed on most
% LaTeX systems. The lastest version and documentation is available at:
% http://www.ctan.org/tex-archive/macros/latex/contrib/mdwtools/


% IEEEtran contains the IEEEeqnarray family of commands that can be used to
% generate multiline equations as well as matrices, tables, etc., of high
% quality.


%\usepackage{eqparbox}
% Also of notable interest is Scott Pakin's eqparbox package for creating
% (automatically sized) equal width boxes - aka "natural width parboxes".
% Available at:
% http://www.ctan.org/tex-archive/macros/latex/contrib/eqparbox/





% *** SUBFIGURE PACKAGES ***
%\usepackage[tight,footnotesize]{subfigure}
% subfigure.sty was written by Steven Douglas Cochran. This package makes it
% easy to put subfigures in your figures. e.g., "Figure 1a and 1b". For IEEE
% work, it is a good idea to load it with the tight package option to reduce
% the amount of white space around the subfigures. subfigure.sty is already
% installed on most LaTeX systems. The latest version and documentation can
% be obtained at:
% http://www.ctan.org/tex-archive/obsolete/macros/latex/contrib/subfigure/
% subfigure.sty has been superceeded by subfig.sty.



%\usepackage[caption=false]{caption}
%\usepackage[font=footnotesize]{subfig}
% subfig.sty, also written by Steven Douglas Cochran, is the modern
% replacement for subfigure.sty. However, subfig.sty requires and
% automatically loads Axel Sommerfeldt's caption.sty which will override
% IEEEtran.cls handling of captions and this will result in nonIEEE style
% figure/table captions. To prevent this problem, be sure and preload
% caption.sty with its "caption=false" package option. This is will preserve
% IEEEtran.cls handing of captions. Version 1.3 (2005/06/28) and later 
% (recommended due to many improvements over 1.2) of subfig.sty supports
% the caption=false option directly:
%\usepackage[caption=false,font=footnotesize]{subfig}
%
% The latest version and documentation can be obtained at:
% http://www.ctan.org/tex-archive/macros/latex/contrib/subfig/
% The latest version and documentation of caption.sty can be obtained at:
% http://www.ctan.org/tex-archive/macros/latex/contrib/caption/




% *** FLOAT PACKAGES ***
%
%\usepackage{fixltx2e}
% fixltx2e, the successor to the earlier fix2col.sty, was written by
% Frank Mittelbach and David Carlisle. This package corrects a few problems
% in the LaTeX2e kernel, the most notable of which is that in current
% LaTeX2e releases, the ordering of single and double column floats is not
% guaranteed to be preserved. Thus, an unpatched LaTeX2e can allow a
% single column figure to be placed prior to an earlier double column
% figure. The latest version and documentation can be found at:
% http://www.ctan.org/tex-archive/macros/latex/base/



%\usepackage{stfloats}
% stfloats.sty was written by Sigitas Tolusis. This package gives LaTeX2e
% the ability to do double column floats at the bottom of the page as well
% as the top. (e.g., "\begin{figure*}[!b]" is not normally possible in
% LaTeX2e). It also provides a command:
%\fnbelowfloat
% to enable the placement of footnotes below bottom floats (the standard
% LaTeX2e kernel puts them above bottom floats). This is an invasive package
% which rewrites many portions of the LaTeX2e float routines. It may not work
% with other packages that modify the LaTeX2e float routines. The latest
% version and documentation can be obtained at:
% http://www.ctan.org/tex-archive/macros/latex/contrib/sttools/
% Documentation is contained in the stfloats.sty comments as well as in the
% presfull.pdf file. Do not use the stfloats baselinefloat ability as IEEE
% does not allow \baselineskip to stretch. Authors submitting work to the
% IEEE should note that IEEE rarely uses double column equations and
% that authors should try to avoid such use. Do not be tempted to use the
% cuted.sty or midfloat.sty packages (also by Sigitas Tolusis) as IEEE does
% not format its papers in such ways.





% *** PDF, URL AND HYPERLINK PACKAGES ***
%
%\usepackage{url}
% url.sty was written by Donald Arseneau. It provides better support for
% handling and breaking URLs. url.sty is already installed on most LaTeX
% systems. The latest version can be obtained at:
% http://www.ctan.org/tex-archive/macros/latex/contrib/misc/
% Read the url.sty source comments for usage information. Basically,
% \url{my_url_here}.





% *** Do not adjust lengths that control margins, column widths, etc. ***
% *** Do not use packages that alter fonts (such as pslatex).         ***
% There should be no need to do such things with IEEEtran.cls V1.6 and later.
% (Unless specifically asked to do so by the journal or conference you plan
% to submit to, of course. )


% correct bad hyphenation here
\hyphenation{op-tical net-works semi-conduc-tor}


\begin{document}
%
% paper title
% can use linebreaks \\ within to get better formatting as desired
\title{Fragmentation: A Comparison of Android Vendor's Bugs via Topic Analysis}


% author names and affiliations
% use a multiple column layout for up to two different
% affiliations

\author{\IEEEauthorblockN{Dan Han, Chenlei Zhang, Xiaochao Fan, Abram Hindle, Kenny Wong and Eleni Stroulia}
\IEEEauthorblockA{Department of Computing Science\\
University of Alberta \\
Edmonton, Canada\\
\{dhan3, chenlei1, xf2, hindle1, kenw, stroulia\}@cs.ualberta.ca}

}

% conference papers do not typically use \thanks and this command
% is locked out in conference mode. If really needed, such as for
% the acknowledgment of grants, issue a \IEEEoverridecommandlockouts
% after \documentclass

% for over three affiliations, or if they all won't fit within the width
% of the page, use this alternative format:
% 
%\author{\IEEEauthorblockN{Michael Shell\IEEEauthorrefmark{1},
%Homer Simpson\IEEEauthorrefmark{2},
%James Kirk\IEEEauthorrefmark{3}, 
%Montgomery Scott\IEEEauthorrefmark{3} and
%Eldon Tyrell\IEEEauthorrefmark{4}}
%\IEEEauthorblockA{\IEEEauthorrefmark{1}School of Electrical and Computer Engineering\\
%Georgia Institute of Technology,
%Atlanta, Georgia 30332--0250\\ Email: see http://www.michaelshell.org/contact.html}
%\IEEEauthorblockA{\IEEEauthorrefmark{2}Twentieth Century Fox, Springfield, USA\\
%Email: homer@thesimpsons.com}
%\IEEEauthorblockA{\IEEEauthorrefmark{3}Starfleet Academy, San Francisco, California 96678-2391\\
%Telephone: (800) 555--1212, Fax: (888) 555--1212}
%\IEEEauthorblockA{\IEEEauthorrefmark{4}Tyrell Inc., 123 Replicant Street, Los Angeles, California 90210--4321}}




% use for special paper notices
%\IEEEspecialpapernotice{(Invited Paper)}




% make the title area
\maketitle


\begin{abstract}

  The fragmentation of the Android ecosystem is a topic of substantial
  debate, and pundits claim (without much actual evidence) that it is
  worsening. In this study, we search for evidence of fragmentation in
  the topics of vendor-specific bug reports. 

  More specifically, we have mined and analyzed the Android bug
  reports related to two popular vendors, HTC and Motorola. First, we
  manually annotated the bug reports with labels. Next, we used
  Labeled-LDA (Latent Dirichlet Allocation) on the labeled data and LDA on the data without manual labels to
  infer their topics. 
  % ES: I have no idea what "By comparing the distribution of average
  % relevance of the top 18 bug topics over time for both vendors"
  % means
  Finally, by examining the relevance of the top 18 bug topics over
  time for the bugs reported by each of the vendors, we categorized the
  topics into three types: \textit{Common Erratic Topics},
  % ES: I propose to make "Common Erratic Topics" into "common problems"
  \textit{Common Inerratic Topics} 
  % ES: I propose to make "Common Impoved Topics" into "common desired
  % improvements  
  % XXX: Fix this. The whole common troubled thing is infuriating
  % wrongly labelled
  and \textit{Unique Topics}. The \textit{Common Erratic Topics} show
  that there is no correlation between the troubled features of
  Android and Android evolution. The \textit{Common Inerratic Topics}
  show that some features within the same vendors have portability
  issues across their multiple devices. The existence of
  \textit{Unique Topics}, exclusive to each vendor, shows that different vendors have specific bug
  topics which imply there may be the portability problem on the
  different vendors.
  % ES: we need a better statement below
  Our findings can be used by Android system community, stakeholders,
  Android device vendors and developers to make project dashboards,
  process investigation and feature analysis.
  In this study we show that there is evidence with Android's bug
  repository of fragmentation across
  Android handsets of multiple vendors.\
  % XXX show we mention the labeled lda results


\end{abstract}

\begin{IEEEkeywords}
Bug reports; Topic mining; LDA; Labeled-LDA; Fragmentation
\end{IEEEkeywords}


% For peer review papers, you can put extra information on the cover
% page as needed:
% \ifCLASSOPTIONpeerreview
% \begin{center} \bfseries EDICS Category: 3-BBND \end{center}
% \fi
%
% For peerreview papers, this IEEEtran command inserts a page break and
% creates the second title. It will be ignored for other modes.
\IEEEpeerreviewmaketitle



\section{Introduction}
% no \IEEEPARstart

Mobile-device vendors continuously compete against each other for
increased market share, and the maket landscape is extremely
volatile\footnote{The Global Smartphone Market Landscape:
  \url{http://www.asymco.com/2011/11/17/the-global-smartphone-market-landscape}
  (retrieved March, 2012)}. Together, iPhone and Android phones
constitute almost 70\% of the US mobile-phone market
\cite{usmarket}. These two very successful platforms are very
different from each other: where Apple tightly controls the software
(iOS) and the hardware (iPhone) platform and its evolution, there
exist a variety of Android phones produced by different vendors, which
often come with custom software, thus giving rise to hardware and
software fragmentation in the Android ecosystem~\cite{analysis}. The
term {\em hardware fragmentation} refers to the fact that at any point
in time, devices based on the same Android operating system run on
different processors, graphics cards, and screen sizes
\cite{analysis}. The term {\em software fragmentation} refers to three
related problems. First, there exist multiple versions of the Android
operating system. Second, vendors offer customized device-specific
Android versions. Finally, carriers also offer software
customizations.

This fragmentation implies an opportunity for personalization and
increased usability, since it enables users to choose the device and
software that best meets their needs. On the flip side, it also
implies that, due to insufficient cross-platform
testing \cite{testing}, Android applications may not behave
consistently across devices and software.  
Fragmentation tends to delay some users from updating their apps
and Android operating system until their specific device is fully supported.
% Furthermore, it tends to
% prevent some Android users from experiencing new features, when these
% features are slow to appear on their chosen devices. 
These problems
may cause users to lose confidence in the Android platform and cost
the brand market-share. Given its potential impact, Android
fragmentation is the topic of much discussion and
controversy. However, there has been little empirical evidence on how
fragmentation impacts consumer opinion.
% Someone from industry performed experiments of Android on different
% devices, and they found that the root cause of fragmentation is the
% classical software engineering issues \cite{testing}.

Our objective in this study is to search for evidence of
Android
fragmentation within the bug reports submitted by users of Android devices
from different vendors. To help explore these bug reports, we applied topic analysis on the
Android bug reports. 
%A topic of the document (e.g. bug reports, source code changes and
%commits) is generated by topic models which has been used to help
%understand software systems. 
%[38] P. Flaherty, G. Giaever, J. Kumm, M. I. Jordan, and
% A. P. Arkin. A latent variable model for chemogenomic
% profiling. Bioinformatics, 21(15):3286, 2005.

% [111] D. Ramage, D. Hall, R. Nallapati, and C. D. Manning. Labeled LDA: a supervised topic model for
% credit attribution in multi-labeled corpora. In Proceedings of the 2009 Conference on Empirical Methods
% in Natural Language Processing: Volume 1-Volume 1, pages 248–256, 2009.
Several topic-analysis methods have been used by researchers in
software engineering, including Latent Dirichlet Allocation (LDA) \cite{Asuncion:2010,Linstead:2009}
, Latent Semantic Indexing (LSI) \cite{Marcus04aninformation}
, and Labeled Latent Dirichlet Allocation
(Labeled-LDA) \cite{labeledlda}. We applied both
Labeled-LDA and LDA topic-analysis on the bug reports
of different vendors, and identified several recurring topics in these
bugs. 
We also compared the results of Labeled-LDA and LDA to see if they
produce similar results even though require very different amounts of
manual effort to use.
Next, we examined the bug topics to gain some intuition on the
causes of Android fragmentation.	

Our study focused on the bug reports of HTC and Motorola, two of the
most prominent Android phone vendors. The first HTC Android phone was
the ``HTC Dream'', manufactured in Oct. 2008. HTC has made more than
thirty different Android phones since then. Motorola produced their
first Android phone in Oct. 2009 and has released more than twenty
different and very popular Android phones since then. 
 
This paper makes three important contributions. First, it provides
empirical evidence about Android features that contribute to both software
and hardware fragmentation. Second, and equally importantly, it lays
out a methodology, that can be used to examine fragmentation within
systems, such as Android, with numerous hardware platforms. Finally,
by applying both Labeled-LDA and LDA we evaluated if the extra effort
to label bug reports for Labeled-LDA is worth the effort compared with
labeling LDA topics.
%Android manufacturers, beyond the two we have already studied. 

The rest of the paper is organized as follows. Section \ref{sec:background} reviews the
background of our work and Section \ref{sec:relatedwork} discusses related work. Section \ref{sec:methodology}
introduces our methodology, which is applied to our data set in
Section \ref{sec:topicanalysis}. Section \ref{sec:fragmentation}
discusses the evidence for fragmentation within Android. Section \ref{sec:comparinglda} compares and evaluates the topic models generated
by LDA and Labeled-LDA. 
%XXX No research questions?
%The paper concludes with a discussion of two
%research questions, 
We conclude with threats to validity in Section \ref{sec:threats},
and conclusion and future work in
Section \ref{sec:conclusions}.

% You must have at least 2 lines in the paragraph with the drop letter
% (should never be an issue)

%\subsection{Subsection Heading Here}
%Subsection text here.


%\subsubsection{Subsubsection Heading Here}
%Subsubsection text here.

\section{Background}
\label{sec:background}


Topic analysis, with respect to {\em Software Control Systems} (SCS) is useful in a variety of text processing applications \cite{hindle9s}. It includes two main steps: topic identification and text segmentation \cite{li2003topic}. It can be used in indexing the texts automatically to retrieve information.
With it, we can understand what the main topics and sets of associated words with these topics, and where those associated words lie within the text \cite{li2003topic}. Recent topic analysis technologies include LDA and Labeled-LDA.

LDA is an unsupervised topic model to credit text documents as mixtures of latent topics, where topics correspond to key word lists presented in the corpus \cite{ldawiki}. It has been successfully used in the software engineering area for mining and retrievng informations from large text corpora.

In our research, we apply Labeled-LDA to perform topic analysis. Labeled-LDA is a supervised topic model for credit attribution in multi-labeled corpora \cite{labeledlda}. It defines a one-to-one mapping between LDA's latent topics and tags labeled by users. In other words, Labeled-LDA incorporates the multiple tags into the topics learning process and only builds topics around these tags, which is quite different from LDA. LDA, as a totally unsupervised algorithm, automatically learns a set of terms for each topic on a corpus without any constraints. To apply Labeled-LDA, we utilize the Stanford Topic Modeling Toolbox (STMT) \cite{stmt}. 

%\begin{figure*}[htb]
%\centering
%\includegraphics[width=1\textwidth]{bugovertime.png}
%\caption{Number of bugs with the major version of Android for HTC and Motorola}
%\end{figure*}



\section{Related Work}
\label{sec:relatedwork}
Topic models have been used to help understand software systems. Marcus et al. \cite{Marcus04aninformation} used Latent Semantic Indexing (LSI) on both source code and user queries and then identified the most relevant source code documents with similarity measurements. Asuncion et al. \cite{Asuncion:2010} applied a coherence measurement on topics learned by LDA to model the quality of bug reports. Linstead et al. \cite{Linstead:2009} performed LDA to generate traceability links for artifacts in software projects automatically. Topic modeling is also utilized by Thomas et al. \cite{Thomas:2011} to study the evolution of topics in software projects.

Compared with all these approaches, the most important difference is the topic models applied in the studies. They used LDA to extract topics, while we applied Labeled-LDA to obtain the topics. With LDA, researchers need to predefine the number of topics and the topics are often hard to interpret. In our work, we first manually labeled bug reports with multiple labels. Then we employed Labeled-LDA which constrains the topic model to just use those topics that correspond to a document's label set \cite{labeledlda}. While during the learning process, LDA cannot incorporate these manual labels. The manual work in our study would overcome the disadvantages of these unsupervised algorithms by pre-defining the number of topics and interpreting the extracted topics. 

\section{Methodology}
\label{sec:methodology}

Our methodology for investigating Android fragmentation starts with extracting
bug reports, labeling the bug reports and then applying
Labeled-LDA on the labeled bug reports.
Then 
 we calculate the average relevance of bug reports to each label
over time \cite{Hindle} and compare them between two Android vendors,
HTC and Motorola in order to look for fragmentation.
We also compare the performance of LDA topics compared with
Labeled-LDA topics by comparing the 
the similarity of each pair of
topics from LDA and Labeled-LDA.

\subsection{Generating the data}

% 1. Extract data: We converted the XML dump of bug reports into SQL
% database. We extracted all the bug reports of HTC and Motorola from
% the SQL database using regular expressions,
% e.g. '%[^0-9a-z]HTC[^0-9a-z]%' and '%[^0-9a-z]motorola[^0-9a-z]%'. We
% also removed the declined and duplicated bug reports.

Our first step was to extract the Android bug reports and then find
the bug reports relevant to HTC and Motorola.  We parse and store the
Android bug reports provided by the MSR Mining
Challenge~\cite{MSRChallenge2012} as table in a SQL Server database.


Then we selected bug reports that identified themselves as being
relevant to HTC or Motorola that mentioned HTC or
Motorola in the title text or the description text of the bug report.  
% We used regular expressions to extract these relevant bug reports
% (e.g.,$'\%[\hat{ }0-9a-z]htc[ \hat{ }0-9a-z]\%'$ and $'\%[ \hat{
% }0-9a-z]motorola[ \hat{ }0-9a-z]\%'$).
%`\texttt{[\^0-9a-z]HTC[\^0-9a-z]}' and
%`\texttt{[\^0-9a-z]motorola[\^0-9a-z]}').
% Using regexes, we extract bug reports highly related to
% Android phones of HTC and Motorola respectively. 
% We use the title,
% opened date and description in each bug report. 
% XXX: What does declined mean: There is a status field in the bug repo. "Declined" is a bug status which means bug reports are not accepted by the bug repo, e.g. the bug reports might not be related to Android devices.
We then removed all the declined and duplicate bug reports, leaving us
with 1503 HTC bug reports and 1058 Motorola bug reports.

\begin{table}[!t]
\renewcommand{\arraystretch}{1.3}
% if using array.sty, it might be a good idea to tweak the value of
% \extrarowheight as needed to properly center the text within the cells
\caption{Manual labels from bug reports of HTC and Motorola.}
\label{selected1}
\centering
\begin{tabular}{|c||l|}
\hline
Vendor & Label\\
\hline
HTC & sms\//mms calling email contact video time network\\ 
  & android\_market display browser bluetooth audio\\
  & notification image SIM\_card settings layout app\\
  & wifi google\_map keyboard calendar alarm language car\\
  & dialing USB touchscreen CPU gtalk voicedialing signal\\
  & google\_voice ringtone google\_navigation location font\\
  & google\_earth battery google\_translate twitter date VPN\\
  & picassa video\_call rSAP region screen\_shot download\\
  & IPV6 SD\_card storage 3G proxy compass calculator\\
  & synchronization  voicemail  voice\_recognition facebook flash\\
  & google\_latitude GPS camera youtube input search radio\\
  & system memory  upgrade  lock\\
\hline
Motorola & calling network settings gtalk calendar signal contact\\
    & android\_market input camera image app wifi keyboard\\
    & layout sms\//mms bluetooth display browser email\\
& alarm audio multimedia\_dock car SD\_card screen\\
& voicedialing battery upgrade dialing ringtone volume\\
& video time swype search exchange headset synchronization\\
& facebook google\_wave download youtube upload\\
& monkey flash VPN touchscreen vibrate CPU system\\
& notification text lock GPS calculator  USB\\
\hline
\end{tabular}
\end{table}


\subsection{Creating Labels and Training Annotators}

% 2. Where are the labels coming from: Before labeling all the bug
% reports, we studied the Android operating system to get the features
% of Android phones
% [http://en.wikipedia.org/wiki/Android_operating_system] and popular
% applications in Android Market
% [https://play.google.com/store/apps]. We also studied the hardware
% components of HTC and Motorola
% [http://en.wikipedia.org/wiki/Comparison_of_Android_devices]. The
% labels are based on these three aspects of Android phones.

In order to investigate fragmentation from a feature-oriented
perspective we needed to label the bug reports by their
relevant features. This would allow us to find feature-relevant bug reports
for each manufacturer.
To ensure our feature-oriented labels would agree with actual
Android features we studied various descriptions~\footnote{Android Operating System summary:
\url{http://en.wikipedia.org/wiki/Android_operating_system};
Android Market: \url{https://play.google.com/store/apps};
Android Comparison:
\url{http://en.wikipedia.org/wiki/Comparison_of_Android_devices}
(retrieved March, 2012).}
 of Android's
operating system, popular apps, and the Android offerings of HTC and
Motorola.

%\subsection{Developing Labels}
%Multi-labeling}

Once we became familiar with the Android operating system and Android
ecosystem we needed to agree and train ourselves to consistently label
Android bug reports.

% 3. Label the bug reports: Zhang and Fan labeled 248 HTC bug reports in
% 2009 separately and then compared the results by each bug to ensure
% the same interpretation of the labels from last step. Then Zhang and
% Fan labeled the rest of the bug reports separately.

%XXX we need a cite
Following a grounded theory-like coding approach, similar to the
approach taken in by Hindle et al.~\cite{Hindle2011}, authors Zhang
and Fan selected a set of HTC 248 bug reports to label
separately. 

To label a bug report, the annotator (Zhang or Fan) reads the bug
report text, both the title and the description, and  then based on their
personal interpretation they related that bug report to the relevant
features. This means that one bug report can receive multiple labels
if it is relevant to multiple identified features. Labels were created
as necessary, if a label regarding a feature did not already exist, it
was created.
These labels 
consisted of the features and applications on an Android mobile
phone, such as SMS/MMS, browser and Wi-Fi or the components of the
handsets mentioned in the bug reports, such as GPS, screens and
keyboards.


To ensure consistency and agreement in labeling the authors trained
themselves in consistent labeling.
Each annotator, authors Zhang and Fan, separately labeled each of these 248 bug reports, 
with labels inspired by the previous research on Android
features. 
Upon completion of labeling, 
 Zhang and Fan compared the labels
and discussed label agreement and
disagreement in order to train themselves to consistently label bug
reports.
The topics of the labeled bug reports were also compared: each annotator's labeled data was used as input to
 Labeled-LDA which produced a set of topics.
These topics and their relevant bug reports were compared to ensure
that annotators had a 
consistent interpretation of the bug reports and their labels.



%First, two of us (Zhang and Fan) separately tag the HTC 248 bug
%reports in 2009 with the multiple labels. 
% Then we train each author’s labeled data with STMT. 
% By comparing each set of trained topics in the results and returning
% to the bug reports, we check to ensure the same interpretation of each
% bug when labeling. 
% Then we come up with the same labeling rules to process the data. 

\subsection{Labeling the HTC and Motorola Bug Reports}

Once the labeling rules were agreed upon each annotator (Zhang and Fan)
seperately labeled HTC and Motorola bug reports, taking over 60 man
hours of manual labeling effort.
%After this, we tag the rest of bug reports separately for both HTC and
%Motorola. 
Using the previously stated labeling methodology, labels were created as necessary.
%When there is a bug report that cannot be tagged by the labels we
%already have, we would create a new label together based on our
%definition of labels, i.e. the functionalities and applications on an
%Android mobile phone or the components of the handsets. 
For example, the label ``calculator'' was created because later in
Android's history there were 
several bug reports about the correctness of the
calculator's results. 
% So we added it to our labels. 
%The manual labeling took approximately 30 hours per person. 


1304 HTC and 985 Motorola bug reports were labeled with multiple
labels, leaving 199 and 73 bug reports that cannot be clearly labeled.
In total, there are 72 labels for HTC and 57 labels for Motorola.
Table \ref{selected1} lists all the manual labels from bug reports of HTC
and Motorola.

\subsection{Applying Labeled-LDA}


% 4. Apply Labeled-LDA:We applied the Labeled-LDA tool, Stanford Topic
% Modeling Toobox [http://nlp.stanford.edu/software/tmt/tmt-0.4/], to
% get the topic-document distribution on our labeled bug reports.

% 5. Calculate the average relevance of each label: The average
% relevance over time of each label is calculated for each vendor. For
% each label the average relevance in one month is the the sum of bug
% reports' relevance divided by the number of bug reports in this
% month. The plot of the average relevance of each label between HTC
% and Motorola is based on this calculation.


Once the bug reports were labeled we wanted to extract the topics
associated with the labels. First we had to process the bug reports 
in order to apply Labeled-LDA to the labeled bug reports. 
We converted the title and description of each bug report to lowercase,
split the text into tokens, and filtered out stop words (words that are less than 3 characters and
common English stop words such as ``all'', ``about'', ``the'',
``that'' and ``were'' ). Then produced word counts/distribution from
these sets of bug-report derived words.

Separately, we applied Labeled-LDA to these processed HTC bug reports and Motrola bug reports.
%We then supplied Labelled-LDA with these word distributions of the bug reports.
Labeled-LDA then outputs the topics, word distributions, associated
with our labels, as well as a document-topic matrix which links our
labels (topics) to the bug reports from HTC and Motorola.
% in the each bug report corpus (HTC and
% Motorola).
%By applying Labeled-LDA to the bug reports of HTC and Motorola
%separately, we have the word distribution of each label and a matrix
%that provides the relationship between bug reports and the labels. 

The topic analysis is based on these results. 
To visualize the association of a label (an extracted
Labeled-LDA topic) to bug reports over time,
%In order to investigate the trend of each label by time, 
we grouped
all the bug reports by month from 2009 to 2011 based on their open
date for each of the two vendors. 
We then computed the average relevance values of bug
reports to this label in each month. 
The average relevance value of a label \begin{math} \l_i \end{math} in
month \begin{math} m_j \end{math} is the sum of all the relevance
values of this label over all bug reports in this month divided by the
number of bug reports in this month,
\begin{equation}
A(\l_i,m_j) = \frac{\sum_{\substack{k=1}}^{|m_j|}r(\l_i,d_k)}{|m_j|}
\label{equation1}
\end{equation}
where $r(\l_i,d_k)$ is the relevance value of label $\l_i$ to bug
report $d_k$, $|m_j|$ is the number of bug reports in this month. 
We generated a distribution of average relevance across three years of
Android history
each label, depicted in Figure \ref{commontopic}, Figure \ref{fixtopic},
Figure \ref{uniquehtc} and Figure \ref{uniquemoto}.


\subsection{Applying LDA}

% 6. Apply LDA: We applied LDA on HTC and Motorola bug reports. We
% tried a set of number of topics for both vendors and in each case we
% tried to label all the topics generated by LDA based on the manual
% labels.

% 7. Choose the number of topics in LDA: We chose the number of topics
% in LDA based the rules that the topics are distinct enough from each
% other, have no repetition and can be well interpreted by us (i.e. we
% can use the manual labels to tag them).


In order to compare the performance between LDA and Labeled-LDA and
 to see if Labeled-LDA is worth the manual labeling effort, 
we applied LDA to the same processed bug reports of HTC and Motorola
but without our manual labels. 

Applying LDA had one complication, LDA requires an input, $n$ that
determines the number of topics that LDA is supposed to extract. 
If $n$ is too large, the topics tend to repeat themselves and tend to
represent similar issues. 
If $n$ is too small, the topics tend to be cluttered and lack a
coherent focus.
This can be interpreted manually by reading the topics and evaluating
the top 10 or 20 words associated with a topic.
To choose the number of topics $n$, we ran LDA using multiple values
of $n$ ranging from 10 to 70, counting by 5,
on the bug reports of HTC. 
Three of the authors (Han, Zhang and Fan) evaluated the word distribution of each topic
together in each case of $n$. 
We determined if topics were distinct enough based on matching the
topics to labels we had created and used for Labeled-LDA. 
%Given
%our previous manual labels that were used by Labeled-LDA we tried to label these LDA topics
%with those labels. 
For a given $n$, if the labels did not repeat too much, and topics did not receive too
many labels, then we preferred that $n$ over others without these
characteristics.
The authors chose $n = 35$, as the topics generated by LDA with $n = 35$
were distinct from each other, had few repetitions and could be
interpreted well by the authors based on their own judgment.
Other researchers had some similar results
\cite{Thomas:2011,Hindle2011,Hindle}. 

We applied the same process to the bug reports of Motorola and we chose
the number of topics to be $n = 30$. 
As described for the HTC bug reports, we also labeled each topics
generated by LDA with our manual labels.
Three of the authors annotated the topics together and it took two
hours in total to finish all the labeling work. 
Table \ref{seleted2} lists a few selected topics from LDA with manual labels.


\subsection{Comparing the Effort to Use LDA and Labeled-LDA}

% 8. Comparison of LDA and Labeled-LDA: For each pair of topics in LDA
% and Labeled-LDA, we computed the their similarity based on the
% topic-document distribution. That is the Jaccard similarity of the two
% sets. One is from LDA and the other one is from Labeled-LDA. Each set
% is the bug reports that have relevance to that label. We chose several
% thresholds on the relevance. That is if the relevance is under the
% threshold, this bug report is not related to that label. At last we
% chose threshold to be 0.2 the mean of the similarities is the
% biggest. 

In order to determine if LDA would generate similar results to
Labeled-LDA we had to compare the topics of each.
% Thus nce LDA and Labelled-LDA were applied to the bug reports of HTC and
% Motorola we had to compare the topics that were extracted.
Both LDA and Labeled-LDA produce matrices of
 the relationships between bug reports of two vendors and the
labels or topics.
Thus we wanted to know if the LDA extracted topics that we manually
labeled matched the Labeled-LDA topics that were labeled via our bug
report labeling. If the results were similar there would be little
point in applying Labeled-LDA in the future.

We determined topic similarity by comparing the sets of documents
relevant to a LDA topic and those relevant to a Labeled-LDA
topic. Because the LDA topic might be different from the Labeled-LDA
topic we did pair-wise similarity comparisons.

We applied the Jaccard similarity coefficient to compute the
similarity between each topic in LDA and each label in Labeled-LDA. 
That is, the Jaccard similarity coefficient between label A in LDA and
label B in Labeled-LDA is the ratio of the intersection of bug reports
related to label A and label B to the union of the bug reports related
to label A and label B,
\begin{equation}
sim(A,B) = \frac{\phi(A,d)\bigcap\phi(B,d)}{\phi(A,d)\bigcup\phi(B,d)}
\end{equation}
where the $\phi(A,d)$ is the set of bug reports that has relevance
values to label A and $d$ is a set of all the bug reports in each
vendor.

The topic-document matrix often contains noise and weak
relationships between topics and documents, thus it is necessary to
provide a threshold of document relevance to determine if a document
is relevant to a topic or not.
We used several thresholds (0.01, 0.05, 0.1, 0.2, 0.3, 0.4 and 0.5) on
the relevance value of a bug report to a topic in LDA when generating
the Jaccard similarity coefficients. 
We eventually chose $0.2$ as the similarities had the biggest mean
value. 
We plotted these pairwise tests (see Figure \ref{similarityhtc} and
Figure \ref{similaritymoto}) in order to explore the match between
LDA and Labeled-LDA.

% XXX WHAT? 
Then we counted the number of bug reports which are related to labels
that are both shared by LDA and Labeled-LDA in HTC and Motorola. 
We applied the Chi-squared test on the two sets of distribution to
study if each of the two distributions match.



\begin{table}[!t]
\renewcommand{\arraystretch}{1.3}
% if using array.sty, it might be a good idea to tweak the value of
% \extrarowheight as needed to properly center the text within the cells
\caption{Selected topics from LDA with manual labels. Word lists are inferred by LDA.}
\label{seleted2}
\centering
\begin{tabular}{|c||c||l|}
\hline
Vendor & Label & Top 10 terms\\
\hline
HTC & sms\//mms &sms, message, text, sent, send, conversation, \\
            && received, reply, time, number \\ \cline{2-3}
  & email & Email, mail, gmail, app. Inbox, send, emails, \\
            &&message, client, read \\ \cline{2-3}
  & browser&browser, page, web, http, open, website, \\
            &&webview, click, url, load\\
\hline
Motorola & wifi &connect, xoom, hotspot, netbook, wifi, ssid, \\
           &&radio, connection, feature, model\\ \cline{2-3}
    &calendar& calendar, event, sync, appointment, date, google, \\
           &&time, droid, day, change \\ \cline{2-3}
    &contact & contact, google, number, address, list, facebook, \\
           &&droid, account, sync, separate \\
\hline
\end{tabular}
\end{table}


\section{Topic Mining and Analysis}
\label{sec:topicanalysis}

In order to investigate fragmentation within Android, we mined the bug reports of Android and analyzed the results from both quantitative and qualitative aspects.

We started by exploring the distribution of the number of bug reports over time for HTC and Motorola. Then we compared and discussed the distribution of average relevance for each topic over time for both vendors.

\begin{figure*}[htb]
\centering
\includegraphics[width=1\textwidth]{bugovertime.png}
\caption{Number of bug reports over time for HTC and Motorola. Bottom horizontal axis is the months from Jan. 2009 to Dec. 2011. The top horizontal axis orders the Android versions by their released date. Vertical axis is the number of bug reports.}
\label{bugovertime}
\end{figure*}

\subsection{Overview of bug reports in HTC and Motorola}

We grouped the bug reports monthly based on their opened date and counted the total number of bug reports in each month for two vendors. Figure \ref{bugovertime} depicts a comparison of the number of bug reports for HTC and Motorola.

From Figure \ref{bugovertime}, we can observe that the first HTC bug report was opened in Jan. 2009, and the first Motorola bug report was opened in Oct. 2009. According to the brief history of Android devices survey \cite{historyofandroid}, HTC released the first Android device in Oct. 2008, while Motorola released its first device in Oct. 2009. The first bug reports of both vendors are in order of the first device released by them. There is a strong time correlation between the first opened bug report and the first released Android device of both vendors.

In addition, Figure \ref{bugovertime} shows the first spike for HTC happened in Sept. 2010, and for Motorola it happened in Dec. 2009. According to counts of bug reports by Android versions, 80\% bug reports are related with Android 2.1 and 2.2. By reading the bug reports, we found that the spike of HTC was caused by the fact that many people upgraded Android version from 2.1 to 2.2, and some features did not work well after upgrade. For example, users could not send message after the upgrade. This suggests that the bug reports activities have strong correlation with Android upgrade. The spike of Motorola mainly contributed to the new features of Android 2.0 and the release of their first Android device which is called Droid (different areas have different product model names, in Europe the model name is A853 or Milestone, in Latin America the model name is A854 or Motoroi). Droid runs Android 2.0 with new features, such as touchscreen display, free turn-by-turn navigation from Google Maps, and sliding QWERTY keyboard. By reading the bug subjects, we found many bug reports related with Google Maps and the sliding QWERTY keyboard. It reveals that there is a high correlation between Android 2.0 and Android Droid.
% xxx [fixed] You didn't compare the hardware platforms. How can you have this conclision?


\subsection{Topics Analysis of HTC and Motorola}

As shown in Table \ref{selected1} we extracted 72 topics for HTC and 57 topics for Motorola with Labeled-LDA.

Based on Equation \ref{equation1} each topic has a distribution of average relevance over time. According to the comparison of each topic's distribution in both vendors, we categorized the topics into three types, which are \textit{Common Erratic Topics}, \textit{Common Inerratic Topics}, and \textit{Unique Topics}. 
The \textit{Common Erratic Topics} mean that the distribution of the average relevance of the topics have fluctuations all the time for both HTC and Motorola. The \textit{Common Inerratic Topics} mean that the distribution of the average relevance of topics turn to be flat over time after several fluctuations for HTC and Motorola. The \textit{Unique Topics} mean that the distribution of average relevance of topics have significant differences between HTC and Motorola.

A representative subset of top 18 topics, which are obtained by
sorting the number of related bug reports for HTC and Motorola
respectively, is given in Table \ref{topicslist}.
Each topic is associated with top 15 terms generated by Labeled-LDA for
both HTC and Motorola. 
%These topics are associated with 85\% bugs of HTC, and 83\% bugs of Motorola. 
As mentioned before, the label column in Table \ref{topicslist} represents the feature of Android. %Please refer to the resarch features as potential labels to have a better understanding of all the labels.

\begin{figure*}[htb]
\centering
\includegraphics[width=1\textwidth]{commontopic.png}
\caption{Common Erratic Topics in HTC and Motorola. X axis is months from Jan. 2009 to Dec. 2011. Y axis is the average relevance of topics.}
\label{commontopic}
\end{figure*}

\subsubsection{Common Erratic Topic} 
%space after Table, Figure
%relevance shoule be written as the distribution of the average relevance
Eight \textit{Common Erratic Topics} shared by two vendors are shown in Table \ref{topicslist} and the distribution of average relevance of each topic is shown in Figure \ref{commontopic}. 
%all the labels should be consistent with all the labels in table I.
\textit{Common Erratic Topics} shown in Table \ref{topicslist} for HTC and Motorola share many identical terms. That means they have the same bug reports about sms\//mms(\textit{text, thread, send}), calendar(\textit{event, day, google,appointment,time}), email(\textit{gmail, send, thread}), contact (\textit{number, google,list}), display (\textit{screen,button,behavior}), bluetooth (\textit{headset,connect, calling}), synchronization (\textit{contact, exchange, google}) and settings(\textit{turn,network,mode}). 

We found that multiple topics share some same terms for each vendor. For HTC, five topics including \textit{sms\//mms, contact, display, bluetooth} and \textit{settings} share the same term ``desire". This indicates that these topics happened frequently in HTC Desire device. \textit{Calendar} and \textit{bluetooth} share the same term ``2.2" and ``2.2" means Android version 2.2. This indicates that these two topics happened frequently for Android 2.2 in HTC devices. For Motorola, seven topics except \textit{settings} share the same term ``droid" and it means Motorola Droid device. In addition, \textit{calendar} and \textit{synchronization} in Motorola share ``milestone" which indicates these two topics discussed mostly in Motorola Milestone device. ``Xoom" shared by \textit{display} and \textit{settings} indicates that Motorola Xoom has more bug reports related with these two topics. Furthermore, \textit{synchronization} associates with both ``Xoom" and ``milestone" terms. This indicates bug reports related with \textit{synchronization} happened frequently in both Motorola Xoom and Motorola Milestone. 

In Figure \ref{commontopic}, HTC and Motorola share the same trend of the distribution of average relevance of topics. Both of them have continuous spikes and drops for each topic over time. There is no obvious decreasing trends of bug reports with Android evolution.
%more explain why android version should make the trend of topics decrease in the ideal case.

In summary, \textit{calendar} in HTC and display in Motorola are strongly correlated with different Android versions. \textit{Bluetooth} in both of HTC and Motorola have strong correlation with Android 2.1 and Android 2.2. With Android evolution, these distribution of average relevance of each topic for both vendors do not demonstrate the decreasing trend with Android evolution as we expect. Both of vendors have some topics associated with their typical devices. For HTC, five out of eight topics have correlation with HTC Desire device. For Motorola, seven out of eight topics have correlation with Motorola Droid device. These eight features which correspond to these topics demonstrate the compatibility issues.

\begin{figure*}[htb]
\centering
\includegraphics[width=1\textwidth]{fixtopic.png}
\caption{Common Inerratic Topics in HTC and Motorola. X axis is months from Jan. 2009 to Dec. 2011. Y axis is the average relevance of topics.}
\label{fixtopic}
\end{figure*}

\subsubsection{Common Inerratic Topic}

Six \textit{Common Inerratic Topics} shared by two vendors are shown in Table \ref{topicslist} and the distribution of the average relevance of each topic is shown in Figure \ref{fixtopic}.

\textit{Common Inerratic Topics} shown in Table \ref{topicslist} for HTC and Motorola share many identical terms for wifi (\textit{connection,ssid,network}), upgrade (\textit{2.2,2.1,http}), and image(\textit{gallery,picture,photo}). Both vendors have the issues in upgrading from Android 2.1 to Android 2.2. This indicates Android 2.2 might have compatibility issue with the upgrading devices. 

Meanwhile, the \textit{Common Inerratic Topics} also own some special terms. For HTC, bug reports related with \textit{calling} happened frequently in Android 2.1, and bug reports related with \textit{image} and \textit{audio} happened frequently in Android 2.2. For Motorola, bug reports related to \textit{calling} happened frequently in Android 2.2. In addition, four out of six topics have correlation with HTC Desire device. For Motorola, seven out of eight topics have correlation with Motorola Droid device and the other one have correlation with Motorola MileStone device. Therefore, these six topics have strong correlation with the Android hardware devices for each vendor.

In Figure \ref{fixtopic}, both HTC and Motorola have spikes in the early stage, and then stay in their values. It indicates the corresponding features of Android tend to be more robust over time with Android evolution during the whole observed period.

In summary, \textit{calling} from both vendors has different correlation with Android versions. With the evolution of Android, these distributions of average relevance of topics demonstrate the improved trends with Android evolution as expected. However, these topics still have strong correlation with their typical devices for both vendors. This indicates these topics contribute to the hardware fragmentation.

% The above summary is not clear to me.

%\begin{figure*}[htb]
%\centering
%\includegraphics[width=1\textwidth]{uniquehtc.png}
%\caption{Unique Topics relevance in HTC}
%\label{uniquehtc}
%\end{figure*}

%\begin{figure*}[htb]
%\centering
%\includegraphics[width=1\textwidth]{uniquemoto.png}
%\caption{Unique Topics relevance in Motorola}
%\label{uniquemoto}
%\end{figure*}


\begin{figure*}[htb]
\centering
\includegraphics[width=1\textwidth]{uniquehtc.png}
\caption{Unique Topics relevance in HTC. X axis is months from Jan. 2009 to Dec. 2011. Y axis is the average relevance of topics.}
\label{uniquehtc}
\end{figure*}

\begin{figure*}[htb]
\centering
\includegraphics[width=1\textwidth]{uniquemoto.png}
\caption{Unique Topics relevance in Motorola. X axis is months from Jan. 2009 to Dec. 2011. Y axis is the average relevance of topics.}
\label{uniquemoto}
\end{figure*}

\begin{figure*}[htb]
\centering
\includegraphics[width=1\textwidth]{htcsim.png}
\caption{Jaccard similarity of labels between LDA and Labeled-LDA in HTC. X axis is the labels in Labeled-LDA and Y axis is the labels of topics generated by LDA. The label ``null" in the Y axis means that topic cannot be labeled. The result is based on the HTC bug reports under the threshold of document relevance of 0.2. Brighter means higher Jaccard similarity.}
\label{similarityhtc}
\end{figure*}

\begin{figure*}[htb]
\centering
\includegraphics[width=1\textwidth]{motosim.png}
\caption{Jaccard similarity of labels between LDA and Labeled-LDA in Motorola. X axis is the labels in Labeled-LDA and Y axis is the labels of topics generated by LDA. The label ``null" in the Y axis means that topic cannot be labeled. The result is based on the Motorola bug reports under the threshold of document relevance of 0.2. Brighter means higher Jaccard similarity.}
\label{similaritymoto}
\end{figure*}


\subsubsection{Unique Topics}

There are two unique topics for HTC shown in Table \ref{topicslist}. Figure \ref{uniquehtc} shows the distribution of the average relevance of each topic.

\textit{HTC Unique Topics} in Table \ref{topicslist} indicates that HTC has an unique topic which is language (\textit{arabic, desire, language, 2.2, letters, characters, translation, character, read}). The associated terms indicate that bug reports related with language happened frequently in Android 2.2. This stems from the fact that the feature of ``keyboard multiple language" is a new feature introduced in Android 2.2. Moreover, most of HTC devices have no physical keyboard, so this new feature has been used frequently by HTC users. In contrast, for Motorola, most of devices have the physical keyboard, so this feature has seldom been used. This fact can also be the reason why HTC has ``on-screen" and ``virtual" terms for \textit{keyboard}, while Motorola does not have these terms at all.

In Figure \ref{uniquehtc}, HTC \textit{keyboard} turns to stay steady, while Motorola has spikes and drops over time. HTC \textit{language} has the relevance distribution, while there are few bug reports related with \textit{language} to make \textit{language} as a topic in Motorola.

There are two unique topics for Motorola shown in Table \ref{topicslist}. Figure \ref{uniquemoto} is a line plot view of the distribution of the average relevance of each topic.

\textit{Motorola Unique Topics} in Table \ref{topicslist} represents HTC and Motorola share the identical terms for GPS (\textit{gps, data, position, location, maps, google, time, lock, wrong, icon, turn, home, latitude}) and browser (\textit{browser, page, text, http, open, server}). Furthmore, both vendors have special terms for topics separately. For browser, Motorola has \textit{droid}, \textit{milestone} (Motorola MileStone is another name of Motorola Droid in different area) and \textit{xoom} terms together. This indicates that the \textit{browser} bug reports happened frequently in two Motorola devices and \textit{browser} has portability issue within Motorola Android devices.

In Figure \ref{uniquemoto}, comparing two vendors, we can see the distribution of the average relevance for \textit{GPS} and \textit{browser} demonstrate different trends. For HTC, they have strikes and drops in the early stage, and then stay steady. For Motorola, they stay steady and then have strikes and drops afterwards. It shows signs of different evolution over the release of Android.  

In summary, for different vendors, they have specific topics which imply there may be portability problems. For a specific vendor, the associated terms of topics implicate that the corresponding features have portability issues across its devices.

\section{Discussion of Fragmentation}
\label{sec:fragmentation}

%What is feature evolution?
According to the analysis about \textit{Common Erratic Topics}, we can see that there is no strong correlation between the feature evolution and Android evolution. In addition, The topic, upgrade has strong correlation with Android 2.1 and Android 2.2. As there are some features evolution demonstrate stable trends with Android evolution implicated by the \textit{Common Inerratic Topics}, we can conclude that Android has compatibility issue in some features.

From \textit{Common Inerratic Topics} and \textit{Unique Topics}, we can see the same topic from different vendors have different correlation, and they have strong correlation with some specific vendors' devices. These observations reveal that Android has portability issue in some features.

When we refer to Android, we generally mean all Android versions existing in the world which include both Android branches from Android community and that from vendors. In the sense of Android itself, we can see that Android has software fragmentation issue. We also discover that there are some features has strong correlation with vendors' devices. In the sense of Android devices from different vendor, we can conclude that Android has hardware fragmentation as well.


\section{Comparing of LDA and Labeled-LDA}
\label{sec:comparinglda}

In this section we investigate if LDA and Labeled-LDA would generate the similar results.

Figure \ref{similarityhtc} and Figure \ref{similaritymoto} depict the pairwise Jaccard similarities of labels from LDA and Labeled-LDA. The brighter spots mean the pair of labels have higher Jaccard similarity. These two labels in LDA and Labeled-LDA would be relevant to more similar set of bug reports. The darker spots mean the pair of labels have lower Jaccard similarity and share less bug reports in common. 

\begin{figure}[htb]
\centering
\includegraphics[width=0.5\textwidth]{htcldallda.png}
\caption{Comparison of number of bug reports related to the same labels from LDA and Labeled-LDA in HTC. The X axis is the same labels from LDA and Labeled-LDA and the Y axis is the number of bug reports.}
\label{bughtc}
\end{figure}

\begin{figure}[!htb]
\centering
\includegraphics[width=0.5\textwidth]{motoldallda.png}
\caption{Comparison of number of bug reports related to the same labels from LDA and Labeled-LDA in Motorola. The X axis is the same labels from LDA and Labeled-LDA and the Y axis is the number of bug reports.}
\label{bugmoto}
\end{figure}


From these two Jaccard similarity plots (Figure \ref{similarityhtc} and Figure \ref{similaritymoto}) of labels between LDA and Labeled-LDA, we can observe that most of the Jaccard similarity values are quite small except a few diagonal ones, especially in HTC. This observation is expected since most of the diagonal spots are the Jaccard similarities between the same labels from LDA and Labeled-LDA. However, even the mean similarities of the diagonal spots are just about 0.2 for HTC and 0.08 for Motorola. The similarity plot for Motorola has much more noise than the plot for HTC.  

%XXX how close
Figure \ref{bughtc} shows the number of bug reports that related to the same labels in the bug reports of HTC and Figure \ref{bugmoto} illustrates the number of bug reports that related to the same labels in the bug reports of Motorola. The $ p $ values of the Chi-squared test on the two sets of distribution are both close to zero. Hence the number of bug reports related to same labels in LDA and Labeled-LDA are quite different.


%Most of the same labels from LDA and Labeled-LDA have the comparable amount of bug reports. For example, the label ``calling'' from the HTC bug reports has exactly the same number of bugs related to for both results of LDA and Labeled-LDA. However, the similarity of these related bug reports in terms of this label ``calling'' is very low which means LDA and Labeled-LDA related quite different bug reports to this label. When doing this comparison, we cannot ignore the number of bugs that related to each label from both two techniques. That is, for one label, the ratio (the smaller number is divided by the bigger number so the ratio is always less or equal to one) of the number of bug reports related to this label predicted by LDA to that of Labeled-LDA would be the upper bound of the similarity value. From Figure \ref{bughtc} and Figure \ref{bugmoto} that the relation between topics and each bug report modeled by LDA is quite different from the results generated by Labeled-LDA.

%The similarity values for these labels in Figure \ref{similaritymoto} are quite low compared with the ratio. Only about ten labels in HTC have similarity values that are larger than half of the ratio. For Motorola, the similarity values are all very low compared with the upper bound of the similarity values.

We can conclude that only few of the bug reports in HTC and Motorola are predicted by LDA and Labeled-LDA to be related to the same labels. In other words, the relation between topics and each bug report modeled by LDA is quite different from the results generated by Labeled-LDA. We think the manual efforts of labeling all the bug reports would help us gain the better topic models generated by Labeled-LDA. 

\begin{table*}[!htb]
\renewcommand{\arraystretch}{1.3}
% if using array.sty, it might be a good idea to tweak the value of
% \extrarowheight as needed to properly center the text within the cells
\caption{Topics and associated Word List with Related Top 15 Terms}
\label{topicslist}
\centering
\begin{tabular}{|c||c||c||c|}
\hline
Topic Type & Label & HTC & Motorola\\ 
\hline
Common Erratic Topics & sms\//mms &message,	sms,	text, thread, time,  & message, text, sms, droid, send,	\\
&& sent, desire, contact, new, number, &	thread, messaging, sent,user, version,\\ 
&&conversation, send, version, app, screen &version, person, threads, number, http\\ \cline{2-4}

  & email & email, mail, gmail, app, message,   &email, droid, account,	gmail, mail, \\
&&inbox, messages,client,emails, account,  &server, message,user,emails, exchange, \\ 
&&send, interface, thread, time, new & file, version, open, device, app\\ \cline{2-4}
            
& calendar&calendar, event, day, events, google,  &calendar,	event, droid, google, appointment, \\
&&view, 2.2,time,month, date, &events, day, field, date, appointments, \\ 
&&version, reminder, appointment,  edit, running &outlook, milestone, data, app, version\\ \cline{2-4}
            
& contact & contact, contacts, number, freed, activity,  &contact, contacts, droid, number, numbers, \\
&&displayed, list, group, google, numbers,   &address, version, google, menu, correct, \\
&&starting,desire, user, version, field & behavior, different,list, option, gmail\\ \cline{2-4}
            
  & display&screen, version, desire,behavior, app, &droid, screen, button, correct, home, \\
&& home, number,code, final, press,  &display, behavior,  landscape, 2.1,  menu, \\
&&sure, user, black, new, power  &bar, xoom,device, user, status\\ \cline{2-4}
   
& bluetooth & bluetooth,headset, car, connect, device,  &bluetooth,	headset, droid, device,connected, \\
 &&connection, version, data, app, desire, & connect, devices, calls,car,issue,	\\
&&desire,	2.2, work, connects, behavior,2.1 & connection, 2.2, car,pair, time\\ \cline{2-4}
            
  & synchronization&contacts, account, sync, exchange, contact, &sync, google, account, contacts, device, \\
&&google, ears, device, group, server, &contact, group, time, exchange, contacts, \\
&&Gmail, policy, new, list, display&display, groups,  list,  droid, milestone\\ \cline{2-4}
            
  & settings&volume, sound,	set, pattern,  default,&settings, device, menu, turn,	network, \\
 && turn, desire, static, control, apps,&vpn, honeycomb, button, xoom,  settings, \\
  && change, settings, media, dns, screen &behavior,	right, wireless, headset, mode\\
\hline

Common Inerratic Topics 
&wifi & wifi, access, network, connection, connect, &wifi, xoom, connect, hotspot, turn, \\
&&router, ssid, desire, http, wi-fi,&connection, ssid, radio, error,signal, \\

&&device, connected, scan, point, app &state, user, time, feature,hotspots\\ \cline{2-4}

&upgrade & update, 2.2, file, 2.1, google  & update, droid, 2.1,2.2, home, \\

 &&version, error, upgrade, froyo, install, & http, version, user, issue, device,\\           
 
 &&work, desire, ota, card, ssl & longer, settings, performance, issues, updated\\            \cline{2-4}
          
&audio& music, audio, player, file, play, &music, droid, player, media, audio,  \\

&&2.2,sound, version, time, playing, & files, volume, play, playing, version,\\

&&playback, app, start, reproduce, mp3 &app, issue, mode, running, genre, sound, user\\ \cline{2-4}
           
&calling& number, calls, calling, 2.1, receive, &droid, calls, number, end, button, \\
&& called, button, answer, bluetooth, desire,  &answer, incoming, screen, voice, speaker, \\ 

&& screen, incoming, works, time, magic & speaker, 2.2, device, place, headphones\\ \cline{2-4}
           
&android market& market, app, google, account, download, &market, apps, app, device, application,  \\

&&update, application, user, device, version, &update, open, user, version, time, \\ 
           
&&apps, paid, desire, installed, application & reporoduce, download, purchase, google, milestone\\ \cline{2-4}   
        
&image & image, gallery, picture, matrix, photo,  &image, droid, wallpaper, gallery, photo,\\
&&null, camera, pictures, version, steps,& picture, device,	file, select,video,\\

&&2.2, photos, code, display, view & folder, load, live, stock, size, screen\\

\hline
HTC Unique Topics & language 
&arabic, desire, language, 2.2, letters, & NONE\\

&&character, translation, character, read, support,& \\

&&sms, write, hebrew, devices,2.3 & \\ \cline{2-4}


& keyboard &keyboard, input,text, key, version,& keyboard, droid, keys,text, press, \\
&& number, typing, on-screen, mode, field, & space, box, open, device, key, \\
&&landscape, virtual, keys, type, message & app, software, 2.0.1, landscape \\
           
           
\hline
Motorola Unique Topics 
& GPS &gps, data, position, location, maps, & maps, gps, google, app, droid, \\

&&google, time,lock, wrong, icon, turn, & location, application, navigation, map,device, \\

&&home, latitude, unit, tag, available & traffic, time, upgrade, turn, route\\ \cline{2-4}

    &browser& browser, page, text, http, open, & browser, droid, page, web, http, open, \\
    &&server,verion, desire, client, web, & xoom, html, behavior, running, links, \\
&& application,2.1, device, button, user  & issue, milestone, 3.1,text \\
\hline
\end{tabular}
\end{table*}

\section{Threats to validity}
\label{sec:threats}

\textit{Construct validity} – Our data originated from MSR Mining Challenge \cite{MSRChallenge2012} and the dataset only ranges from 2009 to 2011. Furthermore we just took all the bug reports related to two vendors in this repository as the dataset to investigate. There may be other bug report repositories can be applied to increase the volume of our dataset. 

\textit{Internal validity} – The explanations and theories we built are based on the actual distributions of all the  average relevance of labels. The trends in the distributions are just manual observations instead of doing statistical analysis. We argue that the differences are distinct enough for us to just do observations. Besides, we might suffer from our bias when choosing the terms generated by Labeled-LDA for each label to do analysis. 

\textit{External validity} – This study focused on only one project since we cannot find an alternative project that was open source project like Android focusing on mobile platform.

\textit{Reliability} – The labels were from the studying features of Android system by two authors (Zhang and Fan). They cannot hide their previous expertise about Android system and handsets. The labels we come up with might suffer from the biased understanding of the aspects in Android system as well as mobile devices. Furthermore, when labeling the bug reports, two annotators followed the same protocol and used the same labels. However, they labeled all the bug reports separately. This might affect the labeling consistency in the dataset. 


\section{Conclusion and Future Work}
\label{sec:conclusions}

In this paper we studied Android bug reports for two vendors, HTC and Motorola. Based on topic analysis using Labeled-LDA on a corpus of manually tagged bug reports with multiple labels, we extracted the top 18 topics and categorized them into \textit{Common Erratic Topics}, \textit{Common Inerratic Topics} and \textit{Unique Topics} for both vendors. The \textit{Common Erratic Topics} show that there is no correlation between the troubled features of Android and Android evolution. In other words, there may be the incompatibility problem existing to the specific features of Android. The \textit{Common Inerratic Topics} show that some features within the same vendors have portability issues across their multiple devices. The \textit{Unique Topics} show that different vendor has specific bug topics which imply there may be the portability problem on the different vendors. Furthermore, we found that the manual efforts of labeling all the bug reports would help us gain the better topic models generated by Labeled-LDA after comparing LDA and Label-LDA. 

For our future work, we will use the name of each hardware model as a label to do topic analysis while applying our methodology in order to discover the effects of different Android versions with respect to compatibility and stability. We will plan to investigate more vendors in order to reveal vendor specific bug topics.

%\subsubsection{Multi-labeling}
%Subsubsection text here.


%\begin{figure}[htb]
%\centering
%\includegraphics[width=0.4\textwidth]{htcldallda.png}
%\caption{Comparison of number of bug reports related to the same labels from LDA and Labeled-LDA in HTC. The X axis is the same labels from LDA and Labeled-LDA and the Y axis is the number of bug reports.}
%\end{figure}
%
%\begin{figure}[htb]
%\centering
%\includegraphics[width=0.4\textwidth]{motoldallda.png}
%\caption{Comparison of number of bug reports related to the same labels from LDA and Labeled-LDA in Motorola. The X axis is the same labels from LDA and Labeled-LDA and the Y axis is the number of bug reports.}
%\end{figure}
%
%\begin{figure}[htb]
%\centering
%\includegraphics[width=0.4\textwidth]{htcratiosim.png}
%\caption{The comparison of ratio and similarity in HTC. The result of the smaller number of bug reports related to this label in LDA or Labeled-LDA divided by the larger one is the ratio of this label. The X axis is the same labels from LDA and Labeled-LDA.}
%\end{figure}
%
%\begin{figure}[htb]
%\centering
%\includegraphics[width=0.4\textwidth]{motoratiosim.png}
%\caption{The comparison of ratio and similarity in Motorola. The result of the smaller number of bug reports related to this label in LDA or Labeled-LDA divided by the larger one is the ratio of this label. The X axis is the same labels from LDA and Labeled-LDA.}
%\end{figure}

% An example of a floating figure using the graphicx package.
% Note that \label must occur AFTER (or within) \caption.
% For figures, \caption should occur after the \includegraphics.
% Note that IEEEtran v1.7 and later has special internal code that
% is designed to preserve the operation of \label within \caption
% even when the captionsoff option is in effect. However, because
% of issues like this, it may be the safest practice to put all your
% \label just after \caption rather than within \caption{}.
%
% Reminder: the "draftcls" or "draftclsnofoot", not "draft", class
% option should be used if it is desired that the figures are to be
% displayed while in draft mode.
%
%\begin{figure}[!t]
%\centering
%\includegraphics[width=2.5in]{myfigure}
% where an .eps filename suffix will be assumed under latex, 
% and a .pdf suffix will be assumed for pdflatex; or what has been declared
% via \DeclareGraphicsExtensions.
%\caption{Simulation Results}
%\label{fig_sim}
%\end{figure}

% Note that IEEE typically puts floats only at the top, even when this
% results in a large percentage of a column being occupied by floats.


% An example of a double column floating figure using two subfigures.
% (The subfig.sty package must be loaded for this to work.)
% The subfigure \label commands are set within each subfloat command, the
% \label for the overall figure must come after \caption.
% \hfil must be used as a separator to get equal spacing.
% The subfigure.sty package works much the same way, except \subfigure is
% used instead of \subfloat.
%
%\begin{figure*}[!t]
%\centerline{\subfloat[Case I]\includegraphics[width=2.5in]{subfigcase1}%
%\label{fig_first_case}}
%\hfil
%\subfloat[Case II]{\includegraphics[width=2.5in]{subfigcase2}%
%\label{fig_second_case}}}
%\caption{Simulation results}
%\label{fig_sim}
%\end{figure*}
%
% Note that often IEEE papers with subfigures do not employ subfigure
% captions (using the optional argument to \subfloat), but instead will
% reference/describe all of them (a), (b), etc., within the main caption.


% An example of a floating table. Note that, for IEEE style tables, the 
% \caption command should come BEFORE the table. Table text will default to
% \footnotesize as IEEE normally uses this smaller font for tables.
% The \label must come after \caption as always.
%
%\begin{table}[!t]
%% increase table row spacing, adjust to taste
%\renewcommand{\arraystretch}{1.3}
%% if using array.sty, it might be a good idea to tweak the value of
%% \extrarowheight as needed to properly center the text within the cells
%\caption{An Example of a Table}
%\label{table_example}
%\centering
%% Some packages, such as MDW tools, offer better commands for making tables
%% than the plain LaTeX2e tabular which is used here.
%\begin{tabular}{|c||c|}
%\hline
%One & Two\\
%\hline
%Three & Four\\
%\hline
%\end{tabular}
%\end{table}


% Note that IEEE does not put floats in the very first column - or typically
% anywhere on the first page for that matter. Also, in-text middle ("here")
% positioning is not used. Most IEEE journals/conferences use top floats
% exclusively. Note that, LaTeX2e, unlike IEEE journals/conferences, places
% footnotes above bottom floats. This can be corrected via the \fnbelowfloat
% command of the stfloats package.


% use section* for acknowledgement
%\section*{Acknowledgment}


% trigger a \newpage just before the given reference
% number - used to balance the columns on the last page
% adjust value as needed - may need to be readjusted if
% the document is modified later
%\IEEEtriggeratref{8}
% The "triggered" command can be changed if desired:
%\IEEEtriggercmd{\enlargethispage{-5in}}

% Better way for balancing the last page:

%\balance

% references section

% can use a bibliography generated by BibTeX as a .bbl file
% BibTeX documentation can be easily obtained at:
% http://www.ctan.org/tex-archive/biblio/bibtex/contrib/doc/
% The IEEEtran BibTeX style support page is at:
% http://www.michaelshell.org/tex/ieeetran/bibtex/
%\bibliographystyle{IEEEtran}
% argument is your BibTeX string definitions and bibliography database(s)
%\bibliography{IEEEabrv,../bib/paper}
%
% <OR> manually copy in the resultant .bbl file
% set second argument of \begin to the number of references
% (used to reserve space for the reference number labels box)
%\begin{thebibliography}{1}

\def\IEEEbibitemsep{0pt plus .5pt}
\small
\bibliographystyle{IEEEtran}
\bibliography{IEEEabrv,msrreference}
%\bibliography{msrreference}




\end{document}
